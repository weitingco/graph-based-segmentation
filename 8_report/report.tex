%Author: Steven Munn, Weiting Ling, Tianchen Jin
%UCSB ECE 278 Final write-up

%HEADERS
\documentclass[12pt, english, titlepage]{article}

\usepackage{seqsplit}
\usepackage{listings}
\usepackage[T1]{fontenc}
\usepackage[latin9]{inputenc}
\usepackage[letterpaper]{geometry}
\usepackage{wrapfig}
\usepackage{subfig}
\usepackage{graphicx}
\usepackage{babel}
\usepackage{amstext}
\usepackage{amsmath}
\usepackage{hyperref}
\usepackage{units}
\usepackage{algorithmicx}
\usepackage{algpseudocode}
\usepackage{amssymb}
\usepackage[ampersand]{easylist}

\usepackage{listings}
\usepackage{color}

\definecolor{dkgreen}{rgb}{0,0.6,0}
\definecolor{gray}{rgb}{0.5,0.5,0.5}
\definecolor{mauve}{rgb}{0.58,0,0.82}

\ListProperties(Hide=100, Hang=true, Progressive=3ex, Style*=$\bullet$ ,
Style2*=$\Rightarrow$)
\lstset{frame=tb,
  language=matlab,
  aboveskip=3mm,
  belowskip=3mm,
  showstringspaces=false,
  columns=flexible,
  basicstyle={\small\ttfamily},
  numbers=none,
  numberstyle=\tiny\color{gray},
  keywordstyle=\color{blue},
  commentstyle=\color{dkgreen},
  stringstyle=\color{mauve},
  breaklines=true,
  breakatwhitespace=true
  tabsize=0
}

\geometry{verbose,tmargin=1in,bmargin=1in,lmargin=1in,rmargin=1in}

\setlength{\parskip}{\smallskipamount}
\setlength{\parindent}{0pt}
\setlength{\parindent}{0.25in}

\newcommand{\tab}{\hspace*{2em}}
%\newcommand{\unit}[1]{\ensuremath{\, \mathrm{#1}}}
\newcommand{\btheta}{\boldsymbol{\theta}}
\newcommand{\p}[1]{\left(#1\right)}
%END HEADERS

\begin{document}

\title{Efficient Graph-based Image Segmentation}

\author{TianChen Jin, Weiting Lin, Steven Munn}

\date{Friday, March 13, 2015}

\maketitle

\section{Questions}

\begin{enumerate}
\item In 2-3 sentences, describe the main project goal.

Our project implements \emph{Efficient Graph-based segmentation} by Felzenszwalb \ref{paper}, first in C/C++, then as a mex wrapper for use with MATLAB, and finally, in Android for a mobile application. Our main goals for the implementation are accuracy and speed relative to the author's original algorithm and implementation. To this end, we plan to perform benchmark tests and comparisons for a quantitative analysis.

\item On a scale of 1 (easy) -10 (impossible), quantify if your original goal was reasonable.

7: the goal presented some challenges, especially for efficient implementation, but it was reasonable for a team of 3. This is why we decided to extend it and add more features. Specifically, we implemented a minimum size for the components, built the nearest-neighbor connected graph in a different color spaces, and experimented with building the graph in a 5-D position and color space.

\item On a scale of 1-100, what is your evaluation of the percent work that you were able to complete, with respect to your initial goal.

120\% with the extensions and experiments we added, we did more than we had set out to do originally.

\item Do you think you were able to achieve the overall base objectives of the project (which is graded for 40% of your course grade) (YES or NO)

Yes, absolutely.

\item Are you asking for EXTRA CREDIT?  (YES or NO)

Yes

\item If your answer to Q5 is YES, please identify (2-3 sentences) additional work that you did beyond the base objectives that deserve the extra credit. You should elaborate this more in your detailed report.

Our extra credit objective was to implement the same algorithm on an Android phone. This involves rewriting code that will work in the context of Android's limited memory availability and library system that isn't compatible with some of the tools we had originally used.

\item Did you write all of the software for your project? (YES or NO).

Yes, except for the benchmark comparisons, obviously.

\item If your answer is NO, please list ALL the sources you used for your project.

http://cs.brown.edu/~pff/segment/

\end{enumerate}


\section{Final Technical Report}


\begin{thebibliography}{2}
\small

\bibitem{paper}
Felzenszwalb, P. F., \& Huttenlocher, D. P. (2004).
	\emph{Efficient graph-based image segmentation}.
	International Journal of Computer Vision, 59(2), 167-181.

\end{thebibliography}
\end{document}